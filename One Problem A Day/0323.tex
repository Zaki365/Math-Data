\documentclass[UTF8]{ctexart}
\title{\textbf{每日一题(4)}}
\date{2019.03.23}
\usepackage{geometry}
\usepackage{amsmath}
\usepackage{amsfonts}
\geometry{left=2.0cm,top=1.5cm,bottom=2.5cm,right=2.0cm}
\begin{document}
\maketitle
设有 $n$ 阶Frobenius阵
\begin{equation*}
F=
  \left( \begin{array}{cccc}
    0 &        &   & -a_n \\
    1 & \ddots &   & \vdots \\
      & \ddots & 0 & -a_2 \\
      &        & 1 & -a_1
  \end{array}\right)
\end{equation*}
求证: $F^n+a_1F^{n-1}+ \cdots + a_nI_n = 0.$
(提示:研究 $F^ie_1, F^ie_2,\cdots,(i=1,2,\cdots,n)$ 其中 $e_j$ 为 $\mathbb{R}^n$ 的第 $j$ 个标准基.)

证: (法一)通过计算,容易得到 $Fe_1=e_2,$ $F^2e_1=F(Fe_1)=Fe_2=e_3,$
于是我们可以归纳出以下结论:
\[
F^je_1=e_{j+1}(j=1,2,\cdots,n-1),
\]

若记 $\alpha=(a_n,a_{n-1},\cdots,a_1)^T,$ 则 $F^ne_1=-\alpha.$ 于是,记 $a_0=0,$ 有:
\[
\left(\sum_{i=0}^{n}a_iF^{n-i}\right)e_1=-\alpha+a_1e_n+a_2e_{n-1}+\cdots+a_ne_1=0.
\]

同理可证 $\displaystyle\left(\sum_{i=0}^{n}a_iF^{n-i}\right)e_j=0$ 对 $j\geq2$ 都成立.

另一方面, 对任意的 $n$ 阶矩阵 $A,$ $A$ 右乘 $e_j$ 得到的向量是 $A$ 的第 $j$ 列元素构成的向量, 于是方阵 $F^n+a_1F^{n-1}+ \cdots + a_nI_n$ 的每一列元素都全为零, 故 $F^n+a_1F^ {n-1}+ \cdots + a_nI_n = 0,$ 结论得证.

(法二:C-H定理)设方阵 $F$ 的特征多项式为 $f(t),$ 由C-H定理可知 $f(F)=0$.

而 $f(t)=\det(tI_n-F)=t^n+a_1t^{n-1}+\cdots+a_n$ (过程略, 递推即可求得), 结合C-H定理即得要证的结论.

\end{document} 