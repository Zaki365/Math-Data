\documentclass[UTF8]{ctexart}

\title{
\textbf{浙江大学 }20 \underline{ 19 } — 20 \underline{ 20 } \textbf{春夏学期} \\
\textbf{《实变函数》课程期末考试试卷}
}

\author{
课程号: \underline{ \quad 751Q0005 \quad }, 开课学院: \underline{ \quad 数学科学学院 \quad } \\
考试试卷: \checkmark A 卷、 B 卷 (请在选定项上打\checkmark) \\
考试形式: \checkmark 闭、开卷 (请在选定项上打\checkmark), 允许带\underline{ \quad 无 \quad  }进场 \\
考试日期: \underline{ \quad 2020 \quad } 年 \underline{ \quad 09 \quad } 月 \underline{ \quad 03 \quad } 日, 考试时间: \underline{ \quad 120 \quad }分钟 \\
\textbf{诚信考试, 沉着应考, 杜绝违纪}
}

\date{}

\usepackage{geometry}
\usepackage{amsmath}
\usepackage{amsfonts}
\usepackage{amssymb}
\usepackage{array}
\usepackage{graphicx}
\usepackage{subfigure}
\usepackage{enumerate}
\pagestyle{plain}
\geometry{top=2.5cm}
\linespread{1.75}
\renewcommand{\d}{\text{d}}

\begin{document}
\maketitle

\begin{center}
考生姓名: \underline{\quad\quad\quad\quad\quad\quad\quad\quad\quad}  学号: \underline{\quad\quad\quad\quad\quad\quad\quad\quad\quad}  所属院系: \underline{\quad\quad\quad\quad\quad\quad\quad\quad\quad}
\end{center}

\centerline{\Large{\textbf{由 CC98 @Serapay 回忆整理}}}

1. (10分) 设 $\mathcal{F}$ 是 $\mathbb{R}$ 上连续函数全体构成的集合, 证明 $\mathcal{F}$ 具有连续统势.

2. (15分) 设可测集 $E$ 满足 $0<m(E)<\infty$, $1\le p_1<p_2<\infty$, 证明 $L^\infty(E)\subset L^{p_2}(E)\subset L^{p_1}(E)$.

3. (20分)
\begin{enumerate}[(1)]
    \item 设可测集 $E\subset\mathbb{R}$, $m(E)>0$, $0<\alpha<1$, 求证: 存在开区间 $I$ 使得 $m(I\cap E)>\alpha\cdot m(I)$.
    \item 设可测集 $E\subset[0,1]$, 若有 $\delta>0$, 使得对 $[0,1]$ 中所有开区间 $(a,b)$ 有 $m(E\cap(a,b))\ge\delta(b-a)$, 求证: $m(E)=1$.
\end{enumerate}

4. (15分) 叙述 Lusin 定理, 并证明: 设 $\{f_k\}_{k\ge1}$ 是 $[a,b]$ 上一列实值可测函数, 则存在正数列 $\{a_k\}_{k\ge1}$ 使得 $a_kf_k(x)\to0$, a.e. $x\in[a,b]$.

5. (15分)
\begin{enumerate}[(1)]
    \item 尽可能多地写出判断一个函数不是绝对连续函数的方法;
    \item 若 $f(x)=\begin{cases}\displaystyle
        x^2\sin\frac{1}{x^2},& x\in(0,1]\\
        0, & x=0
    \end{cases},$ 证明: $f(x)$ 不是绝对连续函数.
\end{enumerate}

6. (10分) 设 $\{f_k\}_{k\ge1}$ 是可测集 $E$ 上的非负可测函数列, 若 $f_k\Rightarrow f$, 证明:
$\displaystyle\int_Ef(x)\d x\le\underset{k\to\infty}{\underline{\lim}}\int_Ef_k(x)\d x$.

7. (15分) 设 $f\in L([0,+\infty))$, 若 $\displaystyle F(x)=\int_0^xf(t)\d t$ 在 $[0,+\infty)$ 上单调增, 求证: $f(x)\ge0$ a.e. $x\in[0,+\infty)$


\end{document}
