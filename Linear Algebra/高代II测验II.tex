\documentclass[UTF8]{ctexart}
\title{\LARGE \textbf{浙江大学 $2018-2019$ 学年春夏学期}}
\author{求是数学班《高等代数 II》测验 II}
\date{2019.05.05}
\usepackage{geometry}
\usepackage{amsmath}
\usepackage{amsfonts}
\geometry{left=2.0cm,top=2.5cm,bottom=2.5cm,right=2.0cm}
\linespread{1.75}
\pagestyle{plain}
\begin{document}
\maketitle
1. 设实对称阵 $\textit{\textbf{A}}\in M_3(\mathbb{R})$ 的全部特征值为 $\lambda_1=\lambda_2=-1,\lambda_3=2$, 且 $x_3=(1,1,1)^T$ 是 $\textit{\textbf{A}}$ 的属于特征值 $2$ 的一个特征向量.

(1) 求 $\textit{\textbf{A}}$ 的属于特征值 $-1$ 的全体特征向量;

(2) 求正交阵 $\textit{\textbf{Q}}$, 使得 $\textit{\textbf{Q}}^T\textit{\textbf{AQ}}$ 为对角阵;

(3) 求 $\textit{\textbf{A}}$.
\\

2. 若 $\alpha_1,\cdots,\alpha_n$ 是 $n$ 维欧式空间 $V$ 的线性无关向量组, 证明: 存在一个向量 $\xi$, 使得 $\langle\alpha_i,\xi\rangle=1(i=1,\cdots,n)$.
\\

3. 设 $T$ 是线性空间 $V$ 上的正规变换, $2,3,6$ 是 $T$ 的三个特征值, 证明: 存在向量 $v\in V$, 使得 $||v||=\sqrt{3},||Tv||=7$.
\\

4. 设 $n$ 为正整数, $T \in L(\mathbb{F}^n)$ 定义如下: $T(z_1,\cdots,z_n)=(0,z_1,\cdots,z_{n-1})$, 求 $T^{*}(z_1,\cdots,z_n)$.
\\

5. 设 $T\in L(V)$ 是自伴随算子, $\lambda\in\mathbb{F},\varepsilon >0$, 证明: 若有 $v\in V$, 使得 $||v||=1$, 且 $||Tv-\lambda v||<\varepsilon$, 则 $T$ 有特征值 $\lambda'$, 使得 $|\lambda-\lambda'|<\varepsilon$.
\\

6. 设 $V$ 是复内积空间, $T\in L(V)$ 是正规算子, 使得 $T^9=T^8$, 证明: $T$ 是自伴随算子, 且 $T^2=T$.
\\

7. 设 $T\in L(V)$ 是自伴随算子, 若 $a,b\in\mathbb{R}$, 满足 $a^2<4b$, 证明: $T^2+aT+bI$ 可逆.
\\

8. 设 $V$ 是 $n$ 维实内积空间, 给定 $V$ 的一个非零向量 $v$, 定义 $H_v:V\rightarrow V$ 如下: 对 $x\in V$,
\[
H_v(x)=x-2\frac{\langle x,v\rangle}{\langle v,v\rangle}v.
\]

(1) 验证: $H_v$ 是正交变换;

(2) 验证: $H_v(v)=-v$, 以及 $H_v(w)=w\Leftrightarrow v\perp w$;

(3) 设 $n\geq3,v_1,\cdots,v_n$ 是 $V$ 的正交基, 证明: 存在实数 $k_1,\cdots,k_n$, 使得 $k_1H_{v_1}+\cdots+k_nH_{v_n}$ 是 $V$ 上的恒等变换.
\\

9. 对线性相关的向量组应用 Gram-Schmidt 过程, 结果会怎样?
\end{document} 