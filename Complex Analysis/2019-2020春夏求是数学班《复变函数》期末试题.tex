\documentclass[UTF8]{ctexart}

\title{
\textbf{浙江大学 }20 \underline{ 19 } — 20 \underline{ 20 } \textbf{春夏学期} \\
\textbf{《复变函数》课程期末考试试卷}
}

\author{
课程号: \underline{ \quad 751Q0006 \quad }, 开课学院: \underline{ \quad 数学科学学院 \quad } \\
考试试卷: \checkmark A 卷、 B 卷 (请在选定项上打\checkmark) \\
考试形式: \checkmark 闭、开卷 (请在选定项上打\checkmark), 允许带\underline{ \quad 无 \quad  }进场 \\
考试日期: \underline{ \quad 2020 \quad } 年 \underline{ \quad 06 \quad } 月 \underline{ \quad 23 \quad } 日, 考试时间: \underline{ \quad 120 \quad }分钟 \\
\textbf{诚信考试, 沉着应考, 杜绝违纪}
}

\date{}

\usepackage{geometry}
\usepackage{amsmath}
\usepackage{amsfonts}
\usepackage{amssymb}
\usepackage{array}
\usepackage{graphicx}
\usepackage{subfigure}
\usepackage{enumerate}
\pagestyle{plain}
\geometry{top=2.5cm}
\linespread{1.75}
\renewcommand{\d}{\text{d}}

\begin{document}
\maketitle

\begin{center}
考生姓名: \underline{\quad\quad\quad\quad\quad\quad\quad\quad\quad}  学号: \underline{\quad\quad\quad\quad\quad\quad\quad\quad\quad}  所属院系: \underline{\quad\quad\quad\quad\quad\quad\quad\quad\quad}
\end{center}

\centerline{\Large{\textbf{\textit{由 CC98 @Serapay 整理}}}}

1. (40分, 每小题 10 分)
\begin{enumerate}[(1)]
\item 求幂级数 $\displaystyle\sum_{n=0}^{\infty}\left[3+(-1)^n\right]^nz^n$ 的收敛半径;
\item 写出一个从第一象限到单位圆盘的双全纯映射 $f$, 满足 $\displaystyle f\left(e^{\frac{\pi i}{4}}\right)=0$
\item 计算积分 $\displaystyle\int_{|z|=1}\frac{e^{\frac{1}{z}}}{z-a}|\d z|$, 其中 $|z|<a$;
\item 将环形区域 $A=\left\{1<|a|<3\right\}$ 上的全纯函数 $\displaystyle f(z)=\frac{1}{z}-\frac{a}{z-b}, a\in\mathbb{C},b\notin A$ 展开成 Laurent 级数.\\
\end{enumerate}

2. (15分) 记 $f(z)=z^d+a_{d-1}z^{d-1}+\cdots+a_1z+a_0$ 是 $d$ 次首一多项式.
\begin{enumerate}[(1)]
  \item 如果当 $|z|\leq 1$ 时, $|f(z)|\leq 1$, 证明: $f(z)=z^d$;
  \item 记 $\displaystyle||f||_\rho=\max\limits_{|z|=\rho}|f(z)|$, 证明: 函数 $\displaystyle h(r)=\frac{||f||_r}{r^d}$ 要么严格单调递减, 要么是常数, 并求出 $h(r)$ 是常数的充要条件.\\
\end{enumerate}

3. (15分) 假设 $f:\mathbb{D}\rightarrow\mathbb{D}$ 全纯.
\begin{enumerate}[(1)]
  \item 证明: $|f(z)-f(-z)|\leq2|z|$;
  \item 若上述不等式对某个 $z_0\neq0$ 等号成立, 那么 $f$ 具有怎么样的表达式? 证明你的结论.\\
\end{enumerate}

4. (20分) 叙述 Riemann 映射定理, 并对有界区域的情形给出证明.
\\

5. (15分) 设 $f$ 在 $\overline{D(0,R)}$ 上全纯, 且在 ${|z|=R}$ 上不取零值. 若
\[
\frac{1}{2\pi i}\int_{|z|=R}\frac{f'(z)}{f(z)}\d z=2,
\]
\[
\frac{1}{2\pi i}\int_{|z|=R}z\frac{f'(z)}{f(z)}\d z=2,
\]
\[
\frac{1}{2\pi i}\int_{|z|=R}z^2\frac{f'(z)}{f(z)}\d z=-4,
\]
求出 $f$ 在 $D(0,R)$ 上的所有根.


\end{document} 