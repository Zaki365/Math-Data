\documentclass[UTF8]{ctexart}

\title{\textbf{浙江大学 $2019 - 2020$ 学年秋冬学期}}
\author{求是数学班《数学分析 III》测验 I}
\date{2019.10}

\usepackage{geometry}
\usepackage{amsmath}
\usepackage{amsfonts}
\usepackage{amssymb}
\usepackage{array}
\usepackage{graphicx}
\usepackage{subfigure}
\usepackage{enumerate}

\linespread{1.75}
\pagestyle{plain}
\geometry{left=2.0cm,top=2.5cm,bottom=2.5cm,right=2.0cm}
\allowdisplaybreaks[4]

\begin{document}
\maketitle
一、求曲面 $S=\left\{(x,y,z)\left| x=u\cos v, y=u\sin v,z=av\right\}\right.$ 在 $\displaystyle\left(0,1,\frac{\pi}{2}a\right)$ 处的切平面, 其中 $a$ 是常数.
\\

二、求 $f(x,y,z)=x-2y+2z$ 在单位球面 $x^2+y^2+z^2=1$ 上的最大值和最小值.
\\

三、叙述无限区间上的广义积分、含参数的广义积分以及数项级数、函数项级数的 Abel 判别法、Dirichlet 判别法并证明数项级数和无限区间上含参数的广义积分的 Abel 判别法. 

(\textbf{Remark: }恶趣味...)
\\

四、证明:
\\

(1) $\displaystyle\int_{0}^{+\infty}\frac{\cos(x^2)}{x^t}{\rm d}x$ 关于 $t$ 收敛但不一致收敛;
\\

(2) $\displaystyle I(t)=\int_{0}^{+\infty}\frac{\cos(x^2)}{x^t}{\rm d}x$ 在 $(-1,1)$ 连续.
\\

五、计算 $\displaystyle I(a)=\int_{0}^{\frac{\pi}{2}}\ln(a^2-\sin^2x){\rm d}x,a>1$.



\end{document}
