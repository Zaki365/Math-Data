\documentclass[UTF8]{ctexart}

\title{\textbf{浙江大学 $2018 - 2019$ 学年秋冬学期}}
\author{求是数学班《数学分析 I》测验 I}
\date{2018.10}

\usepackage{geometry}
\usepackage{amsmath}
\usepackage{amsfonts}
\usepackage{amssymb}
\usepackage{array}
\usepackage{graphicx}
\usepackage{subfigure}
\usepackage{enumerate}

\linespread{1.75}
\pagestyle{plain}
\geometry{left=2.0cm,top=2.5cm,bottom=2.5cm,right=2.0cm}
\allowdisplaybreaks[4]

\begin{document}
\maketitle
一、判断下面陈述是否正确, 错误的请给出反例.
\\

(1) 数列 $\{x_n\}$ 无界当且仅当存在子列 $\displaystyle\left\{x_{n_k}\right\}$, 使得 $\displaystyle\lim\limits_{k\rightarrow\infty}x_{n_k}=\infty$;
\\

(2) 设函数 $f(x)$ 在 $[a,b]$ 上无界, 则存在 $x_0\in[a,b]$ 使得 $\displaystyle\lim\limits_{x\rightarrow x_0}f(x)=\infty$;
\\

(3) 设 $f(x),g(x)$ 当 $x\rightarrow x_0$ 时是无穷大量, 则 $f(x)+g(x)$ 当 $x\rightarrow x_0$ 时也是无穷大量;
\\

(4) 有界数列至多有有限多个聚点.
\\

二、叙述并证明函数极限存在的 Cauchy 收敛准则.
\\

三、计算下列极限.

(1) $\displaystyle\lim\limits_{n\rightarrow\infty}\sqrt[n]{2+\sin n}$;
\\

(2) $\displaystyle\lim\limits_{n\rightarrow\infty}\frac{\sum\limits_{k=1}^{n}\frac{1}{\sqrt{2k-1}}}{\sqrt{n}}$;
\\

(3) $\displaystyle\lim\limits_{n\rightarrow\infty}\frac{1-\cos\frac{1}{n}}{\ln(1+\frac{1}{n^2})}$;
\\

(4) $\displaystyle\lim\limits_{n\rightarrow\infty}\left(1-\frac{1}{n^2}\right)^n$.
\\

四、设 $\displaystyle x_n=\sum\limits_{k=1}^{n}\frac{(-1)^{k-1}}{k},n\geq1$. 证明 $\{x_n\}$ 收敛并求其极限.
\\

五、设 $f(x),g(x)$ 在 $x_0$ 的某个邻域内有界, 证明:
\\

(1) $\displaystyle\underset{x\rightarrow x_0}{\overline{\lim }}\left(f(x)+g(x)\right)\geq\underset{x\rightarrow x_0}{\overline{\lim }}f(x)+\underset{x\rightarrow x_0}{\underline{\lim }}g(x)$;
\\

(2) 若 $\displaystyle\lim\limits_{x\rightarrow x_0}f(x)$ 存在, 则 $\displaystyle\underset{x\rightarrow x_0}{\overline{\lim }}\left(f(x)+g(x)\right)=\lim\limits_{x\rightarrow x_0}f(x)+\underset{x\rightarrow x_0}{\underline{\lim }}g(x)$.
\\

六、设 $\{x_n\}$ 有界且 $\displaystyle\lim\limits_{n\rightarrow\infty}(x_n-2x_{n+1}+x_{n+2})=0$, 证明: $\displaystyle\lim\limits_{n\rightarrow\infty}(x_n-x_{n+1})=0$.


\end{document}
