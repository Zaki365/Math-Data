\documentclass[UTF8]{ctexart}

\title{\textbf{浙江大学 $2019 - 2020$ 学年秋冬学期}}
\author{求是数学班《数学分析 III》期中}
\date{2019.11}

\usepackage{geometry}
\usepackage{amsmath}
\usepackage{amsfonts}
\usepackage{amssymb}
\usepackage{array}
\usepackage{graphicx}
\usepackage{subfigure}
\usepackage{enumerate}

\linespread{1.75}
\pagestyle{plain}
\geometry{left=2.0cm,top=2.5cm,bottom=2.5cm,right=2.0cm}
\allowdisplaybreaks[4]

\begin{document}
\maketitle
一、设常数 $a>0$, 求曲线
\[
\begin{cases}
x^2+y^2=2az \\
x^2+y^2+xy=a^2 
\end{cases}
\]
上的点到 $xy$ 平面的最大距离和最小距离.
\\

二、证明 $\displaystyle I(\lambda)=\int_{1}^{+\infty}\frac{\sin(\lambda x)}{x}{\rm d}x$ 在 $(0,+\infty)$ 可导.
\\

三、计算积分 $\displaystyle I=\iiint\limits_{\Omega}z{\rm d}x{\rm d}y{\rm d}z$, 其中 $\Omega$ 是球体 $x^2+y^2+z^2\leq R^2$ 和 $x^2+y^2+z^2\leq2Rz$ 的公共部分, $R>0$ 是常数.
\\

四、判断下列广义重积分的敛散性并说明理由.
\\

(1) $\displaystyle\iint\limits_{\mathbb{R}^2}\sin(x^2+y^2){\rm d}x{\rm d}y$;
\\

(2) $\displaystyle\iint\limits_{0\leq x,y\leq1}\frac{{\rm d}x{\rm d}y}{x+y}$.
\\

五、计算 $\displaystyle I=\iint\limits_{\mathbb{R}^2}e^{-(x^2+2pxy+y^2)}{\rm d}x{\rm d}y$, 其中 $p\in(0,1)$ 是常数.
\\

六、已知螺旋面 $S=\{(x,y,z)|x=r\cos\theta, y=r\sin\theta, z=h\theta\}$, 其中 $h>0$ 是常数. 求曲面 $S$ 在 $0\leq r\leq a,0\leq\theta\leq2\pi$ 部分的面积.
\\

七、设曲线 $\Gamma:x=x(t),y=y(t)(t\in[\alpha,\beta])$ 在 $\mathbb{R}^2$ 中连续且 $x(t)\in C^1[\alpha,\beta]$. 证明: $J(\Gamma)=0$.
\\

八、设 $D\subset\mathbb{R}^d$ 是 Jordan 可测紧子集, $f$ 是 $D$ 上有界函数, 记 $X$ 为 $f$ 的不连续点集. 证明: $f$ 在 $D$ 可积当且仅当 $X$ 是 Lebesgue 零测集.

\end{document}
