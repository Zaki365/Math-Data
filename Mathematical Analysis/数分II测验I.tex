\documentclass[UTF8]{ctexart}
\title{\LARGE \textbf{浙江大学 $2018-2019$ 学年春夏学期}}
\author{求是数学班《数学分析 II》测验 I}
\date{2019.03.26}

\usepackage{geometry}
\usepackage{amsmath}
\usepackage{amsfonts}
\usepackage{amssymb}
\usepackage{array}
\usepackage{graphicx}
\usepackage{subfigure}
\usepackage{enumerate}

\linespread{1.75}
\pagestyle{plain}
\geometry{left=2.0cm,top=2.5cm,bottom=2.5cm,right=2.0cm}
\allowdisplaybreaks[4]

\begin{document}
\maketitle
一、求极限:
\\

(1) $\lim \limits_{x \rightarrow 0} \displaystyle \frac{\int_{0}^{x}\sin (x+t)^2 {\rm d}t}{x^3};$
\\

(2) $\lim \limits_{n\rightarrow\infty} \displaystyle \frac{\sqrt[n]{n(n+1)\cdots(2n-1)}}{n}.$
\\

二、求积分:
\\

(1) $\displaystyle \int_{0}^{a}x^2\sqrt{a^2-x^2}{\rm d}x,$ 其中 $a>0$ 是常数;
\\

(2) $\displaystyle \int_{0}^{1}x\arctan x{\rm d}x;$
\\

(3) $\displaystyle \int_{1}^{e}\sin(\ln x){\rm d}x;$
\\

(4) $\displaystyle \int_{0}^{\frac{\pi}{2}}\displaystyle \frac{x}{\sin x +\cos x}{\rm d}x.$
\\

三、求曲线 $x^{\frac{2}{3}}+y^{\frac{2}{3}}=1$ 的长度.
\\

四、设 $f$ 在 $[0,1]$ 上 Riemann 可积, 证明 $\lim \limits_{n\rightarrow\infty}\displaystyle \int_{0}^{1}x^nf(x){\rm d}x=0.$
\\

五、叙述有界函数的 Darboux 积分理论.
\\

六、设 $f\in C^1([0,1]), f(0)=0,\displaystyle \int_{0}^{1}(f'(x))^2{\rm d}x=1.$ 证明: $\max \limits_{x \in [0,1]}|f(x)|\leq 1,$ 等号成立当且仅当 $f(x)\equiv x$ 或 $f(x)\equiv -x.$
\\

七、设 $f \in R([a,b])$ 且 $f(x)$ 几乎处处为零, 即存在零测度集 $E \subset [a,b],$ 使得 $f|_{[a,b]\setminus E}\equiv0.$ 证明: $\displaystyle \int_{a}^{b}f(x){\rm d}x=0.$
\end{document} 